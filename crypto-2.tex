\documentclass[11pt,a4paper]{book}
\usepackage[english,greek]{babel}
\usepackage[utf8x]{inputenc}
\usepackage{amssymb,latexsym,amsmath,ucs,amsthm,setspace,graphicx}
\newtheorem*{lemma}{Λήμμα}
\newcommand{\HRule}{\rule{\linewidth}{0.5mm}}
\newcommand{\defeq}{\overset{\underset{\mathrm{def}}{}}{=}}
\makeatletter
\def\imod#1{\allowbreak\mkern10mu({\operator@font mod}\,\,#1)}
\makeatother
\raggedbottom
\begin{document}

\begin{titlepage}
\begin{center}

\includegraphics[width=0.15\textwidth]{Pyrforos2.png}\\[1.cm]
\textsc{\LARGE Εθνικό Μετσόβιο Πολυτεχνείο}\\[1.5cm]

\Large{ Σημειώσεις Διαλέξεων }\\[0.5cm]

% Title
\begin{doublespace}
\HRule \\[0.4cm]
{\huge \bfseries
Στοιχεία Θεωρίας Αριθμών
\\
\&
\\
Εφαρμογές στην Κρυπτογραφία
}\\[0.4cm]
\end{doublespace}

\HRule \\[1.5cm]

\begin{minipage}{0.4\textwidth}
\begin{flushleft} \large
\emph{Επιμέλεια σημειώσεων:}\\
Διονύσης \textsc{Ζήνδρος}
Αντώνης \textsc{Αναστασόπουλος}
\end{flushleft}
\end{minipage}
\begin{minipage}{0.4\textwidth}
\begin{flushright} \large
\emph{Διδάσκοντες:} \\
Στάθης \textsc{Ζάχος}\\
Άρης \textsc{Παγουρτζής}
\end{flushright}
\end{minipage}

\vfill

{\large \today}
\end{center}
\end{titlepage}

\section*{Επίλυση γραμμικής ισοτιμίας}

Ζητείται να λυθεί η ισοτιμία ως προς $x$:
\[
ax \equiv b \imod n
\]

Για παράδειγμα:

\[
4x \equiv 6 \imod 7
\]

Καθώς βρισκόμαστε σε αριθμητική \textlatin{modulo}, μπορούμε να δοκιμάσουμε εξαντλητικά τις πιθανές λύσεις:

\[
4 \mathbb{Z}_7 = \{ 0, 4, 1, 5, 2, 6, 3 \}
\]

Πράγματι η μοναδική λύση της ισοτιμίας είναι:

\[
x \equiv 5 \imod 7
\]

\begin{lemma}
Η ισοτιμία $ax \equiv b \imod n$ έχει μοναδική λύση όταν $\gcd( a, n ) = 1$.
\end{lemma}
\begin{proof}
Πράγματι, υπάρχει η λύση $x \equiv a^{-1}b \imod n$. Έστω τώρα ότι υπάρχουν λύσεις $x$, $x'$. Τότε έχουμε:

\begin{align*}
	\left\{ \begin{array}{ll}
         ax &\equiv b \imod n\\
         ax' &\equiv b \imod n\end{array} \right.
    &\Rightarrow ax \equiv ax' \imod n\\
    &\Leftrightarrow n|a(x - x')\\
    &\Leftrightarrow n|(x - x')\\
    &\Leftrightarrow x \equiv x' \imod n
\end{align*}
\end{proof}

Το αποτέλεσμα αυτό δεν ισχύει απαραίτητα όταν $\gcd( a, n ) \neq 1$. Για παράδειγμα η εξής ισοτιμία δεν έχει λύση:

\[
10x \equiv 6 \imod{35}
\]

Πράγματι, θα είναι:
\[
10\mathbb{Z}_{35} = \{ 0, 10, 20, 30, 5, 15, 25, 0, 10, 20, 30, \dots \}
\]

Ενώ για παράδειγμα η εξής ισοτιμία έχει πολλές λύσεις:
\[
10x \equiv 30 \imod{35}
\]

\begin{lemma}
Έστω $d = \gcd( a, n ) > 1$. Η ισοτιμία $ax \equiv b \imod n$ έχει ακριβώς $d$ λύσεις ανν $d|b$. Διαφορετικά δεν έχει λύσεις.
\end{lemma}
\begin{proof}
Θα δείξουμε ότι $\exists x \Rightarrow d|b$. Πράγματι:
\begin{align*}
& ax \equiv b \imod n \\
\Leftrightarrow & n | (ax - b)\\
\Rightarrow & d|(ax-b)\\
\Rightarrow & d|b
\end{align*}
Αντίστροφα έχουμε:
\begin{align*}
& ax \equiv b \imod n\\
\Leftrightarrow & n | (ax - b)\\
\Leftrightarrow & \exists \lambda \in \mathbb{Z}: ax - b = \lambda n\\
\Leftrightarrow & \exists \lambda \in \mathbb{Z}: \frac{a}{d}x - \frac{b}{d} = \lambda\frac{n}{d} \\
\Leftrightarrow & \frac{n}{d} | (\frac{a}{d}x - \frac{b}{a}) \\
\Leftrightarrow & \frac{a}{d}x \equiv \frac{a}{d}x \equiv \frac {b}{d} \imod{\frac{n}{d}}
\end{align*}
Άρα η μοναδική λύση στο $\mathbb{Z}_{\frac{n}{d}}$ θα είναι η:
\[
x_0 \equiv {(\frac{a}{d})}^{-1}_{\imod {\frac{n}{d}}} \frac{b}{d} \imod{\frac{n}{d}}
\]
Συνεπώς οι λύσεις στο $\mathbb{Z}_n$ θα είναι οι εξής $d$:
\[
\{x_0 + i\frac{n}{d}: 0 \leq i < d \}
\]
\end{proof}

\begin{lemma}
Απαλοιφής
\[
ax \equiv ax' \imod n \Leftrightarrow x \equiv x' \imod {\frac{n}{\gcd(a, n)}}
\]
\end{lemma}

\section*{Επίλυση τετραγωνικής ισοτιμίας}

Ζητείται να λυθεί η ισοτιμία ως προς $x$:
\[
ax^2 \equiv b \imod n
\]

Θα εξετάσουμε την περιπτωση:
\[
x^2 \equiv b \imod n
\]

\begin{lemma}
Έστω ότι $n = p$ πρώτος. Τότε αν η ισοτιμία $x^2 \equiv b \imod n$ έχει λύσεις, τότε αυτές είναι ακριβώς $2$ και μεταξύ τους αντίθετες.
\end{lemma}
\begin{proof}
Έστω $x$ και $y$ δύο διαφορετικές λύσεις της ισοτιμίας.
\begin{align*}
&x^2 \equiv y^2 \imod p\\
\Leftrightarrow & p | (x-y)(x+y)\\
\Leftrightarrow & p | (x-y) \lor p | (x+y)\\
\Leftrightarrow & x \equiv y \imod p \lor x \equiv -y \imod p
\end{align*}
\end{proof}

\begin{lemma}
Έστω ότι $n$ γινόμενο δύο πρώτων $n = pq$. Τότε αν η ισοτιμία $x^2 \equiv b \imod n$ έχει λύσεις, τότε αυτές είναι ακριβώς $2$ ή $4$ και θα είναι μεταξύ τους ανά 2 αντίθετες.
\end{lemma}
\begin{proof}
Έστω λύσεις της ισοτιμίας $x$ και $y$. Τότε παρόμοια με παραπάνω θα έχουμε:
\begin{align*}
\left\{\begin{array}{llll}
     & p|(x-y) \land q|(x+y)\\
\lor & p|(x-y) \land q|(x-y)\\
\lor & p|(x+y) \land q|(x+y)\\
\lor & p|(x+y) \land q|(x-y)
\end{array}\right.\\
\Leftrightarrow \left\{\begin{array}{ll}
     & x \equiv \pm y \imod p\\
     \land x \equiv \pm y \imod q
\end{array}\right.
\end{align*}
\end{proof}

Για παραδειγμα:

\begin{align*}
	&x^2 \equiv 29 \imod {35}\\
	\Rightarrow &
	\left\{\begin{array}{ll} & x^2 \equiv 29 \imod 5\\
	& x^2 \equiv 29 \imod 7\end{array}\right.\\
	\Rightarrow &
	\left\{\begin{array}{ll} & x^2 \equiv 4 \imod 5\\
	& x^2 \equiv 1 \imod 7\end{array}\right.\\
	\Rightarrow &
	\left\{\begin{array}{ll} & x \equiv \pm 2 \imod 5\\
	& x \equiv \pm 1 \imod 7\end{array}\right.\\
	\Rightarrow &
	x \equiv \left\{\begin{array}{llll} 22\\27\\8\\13 \end{array}\right. \imod {35}
\end{align*}

Στα παραπάνω θεωρήματα η λύση $0$ θεωρείται διπλή.

\section*{Ασκήσεις}
\begin{enumerate}
	\item Να αποδειχθεί το Λήμμα Απαλοιφής.
\end{enumerate}

% \section*{Βιβλιογραφία}
% \begin{enumerate}
% 	\item \text{[}Ζάχος\text{]}: Ε. Ζάχος, ((Σημειώσεις στη Θεωρία Αριθμων και την Κρυπτογραφία)), 2007
% \end{enumerate}
\end{document}
